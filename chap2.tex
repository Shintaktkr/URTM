
\chapter{関連研究}
\section{可逆計算}
可逆計算とは,計算過程において,一度の状態においても直前と直後にとり得る状態
を高々一つもつものである.$Landauer$は,計算機において非可逆な演算はエネルギーの放出を伴うことを指摘した.可逆的な演算はこのような不可避なエネルギーの放出を減らす1つの解
決策として用いられる.可逆計算では,入力から出力までの過程を出力から逆算
して入力を求めることが可能であるため,情報を消失することなく出力結果を導
くことが出来る.

\section{可逆プログラミング言語}
可逆プログラミング言語とは,そのプログラムの実行過程が必ず可逆になるよう
な言語設計がなされているプログラミング言語である.可逆プログラミング言語
の例を挙げると,$Janus$,$R$などが存在する.可逆プログラミング言語は,可
逆であることの形式的証明されている.それらの言語は非可逆なプログラム
を記述することができないため,可逆プログラミング言語で書かれた任意のプログラム
の可逆性が保証される.

 \subsection{Janus}
$Janus$とは可逆プログラミング言語の一種で,多重集合と配列の書換えに基づく制約処理モデルを持つ.可逆プログラミング言語である$Janus$は,C言語に似た構文に加えて可逆性を保
証するための構文規則を持つ.
以下大久保くんのを参考

 \subsection{WHILE言語}
\texttt{WHILE}言語とは,ローカル変数,$if$文,$while$文, $loop$文,およ
び単純な整数と$boolean$型の代入を伴う簡単な命令型言語である.
構文規則は以下のとおりである.\\
\begin{figure}[t]
\begin{alltt}
S \(::=\) x:=a                         命令
    |     skip
    |     S_(1);S_(2)
    |     if b then S_(1) else S_(2)
    |     while b do S
a \(::=\) x               変数式
    |     n
    |     a_(1) op_(a) a_(2)
op_(a) \(::=\) + | − | ∗ | / | ...  算術演算子
b \(::=\) true              論理型式
    | false 
    | not b
    | b_(1) op_(b)  b_(2)
    | a_(1) op_(r)  a_(2)
op_(b) \(::=\) and | or | * | / | ... 倫理演算子
op_(r) \(::=\) < | ≤ | = | > | ≥ | ... 補助演算子
\end{alltt}
\end{figure}
 \subsection{セルオートマトン}
セルオートマトンとは1950年代に von bNeumann(ジョン・フォン・ノイマン) が自己増殖オートマトンを設計するための理論的枠組みとして提案されたモデルである.
いくつかの状態を持つセルという単位によって構成され,事前に設定された規則に従い,そのセル自身や近傍の状態によって,時間発展と共に,その時間における各セルの状態が決定される.
現在では,計算システムの数理モデルの一種として扱われ,計算の基礎理論をは
じめとして,交通流や生物などのシステムのシミュレーションに用いられている.
	標準的なCAは
	\[
 	(\mathbb{Z}^k,Q,N,f,\#)
	\]
	として定める.
	ただし,$Q$はセルの状態と呼ばれる空でない有限集合,
	$N(=\{n_{1},\ldots,n_{m}\})$は近傍と呼ばれる$\mathbb{Z}^k$の部分
	集合,局所関数$f:Q^m \rightarrow Q$は各セルの状態遷移を定めるも
	のとする.
	なお,$\#\in Q$は静止状態と呼ばれ,$f(\#,\ldots,\#)=\#$を満たす.
	集合$Q$上の$k$次元の状相とは$\alpha : \mathbb{Z}^k \rightarrow Q$であるような写像$\alpha$をいう.
	$Q$上の$k$次元状相すべての集合を$\mathrm{Conf}_k(Q)=\{\alpha \mid \alpha : \mathbb{Z}^k \rightarrow Q\}$で表す.
	状相$\alpha$は,集合$\{x\mid x\in \mathbb{Z}^k \wedge \alpha (x) \neq \#\}$が有限であるとき,有限と呼ばれる.\par
	大域関数$F: \mathrm{Conf}_k(Q) \rightarrow \mathrm{Conf}_k(Q)$を
	%
	\begin{eqnarray}
 	\forall \alpha \in \mathrm{Conf}_k(Q),\forall x \in \mathbb{Z}^k: \hspace{60pt} \nonumber \\
	~~F(\alpha)(x) = f(\alpha(x+n_1),\ldots,\alpha(x+n_m))
	\end{eqnarray}
	%
	と定める.

\subsubsection{可逆セルオートマトン}
可逆セルオートマトンとは,可逆性が保証されたセルオートマトン,つまり,現
在の状態から1つ前の状態を一意に決めることができるセルオートマトンのこと
である.可逆セルオートマトンは,以下の定義を満たしている.
\begin{enumerate}
\item 大域関数$F$が単射であること.
\item 状相の遷移がちょうど逆であるようなセルオートマトンが存在すること.
\end{enumerate}

1次元,2次元可逆セルオートマトンは計算可逆万能性をもつモデルが数多く発見されている.
